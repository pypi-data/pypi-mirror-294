\documentclass{article}
\usepackage{graphicx}

\begin{document}

\title{JupyterLab \LaTeX}
\date{}
\maketitle

\section{Introduction}
This is a sample document demonstrating the ability to live-edit
\LaTeX documents in JupyterLab.
\\
\\
Right-click on this document, and select "Show LaTeX Preview".
A new panel should open up on the right with the PDF that is generated
by this document. If there are any errors in the document, an
error panel should open below.
\\
\\
We can write equations:
\begin{equation}
    \rho \left( \frac{\partial \mathbf{u}}{\partial t} + \mathbf{u} \cdot \nabla \mathbf{u} \right) =
    -\nabla P + \eta \nabla^2 \mathbf{u} + \rho \mathbf{g}.
\label{eq:Navier–Stokes}
\end{equation}
\\
\\
And we can write tables:
\begin{center}
  \begin{tabular}{ | l | c | r| }
    \hline
    1 & 2 & 3 \\ 
    4 & 5 & 6 \\ 
    7 & 8 & 9 \\
    \hline
  \end{tabular}
\end{center}

We can reference equations by their numbers, i.e. Equation (\ref{eq:Navier–Stokes}).

\subsection{We can add new subsections}
And we can include images, such as the Jupyter logo:
\begin{center}
  \includegraphics[width=0.65\textwidth]{images/jupyter_logo.png}
\end{center}



\pagebreak




You can also use SyncTeX by typing ⌘(CMD)/CTRL+⇧(SHIFT)+X.

\end{document}
